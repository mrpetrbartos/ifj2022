\documentclass[a4paper,12pt]{article}
\usepackage[utf8]{inputenc}
\usepackage[IL2]{fontenc}
\usepackage[left=1.5cm,top=2.5cm,text={18cm,25cm}]{geometry}
\usepackage[czech]{babel}
\usepackage{times}
\usepackage{enumitem}
\usepackage{array,etoolbox}
\preto\tabular{\setcounter{rule_table_number}{0}}
\newcounter{rule_table_number}
\newcommand\rownumber{\stepcounter{rule_table_number}\arabic{rule_table_number}}

\begin{document}
\begin{titlepage}
	\begin{center}
		\textsc{{\Huge Vysoké učení technické v~Brně\\[0.4em]}}
		\textsc{{\huge Fakulta informačních technologií}}
		\vspace{\stretch{0.382}}
				    
		{\Large Dokumentace projektu z předmětů IFJ a IAL\\[0.3em]}
		\textbf{{\LARGE Implementace překladače imperativního jazyka IFJ22\\[0.4em]}}
		{\Large Tým xbarto0g, varianta\,--\,TRP}
		\vspace{\stretch{0.618}}
				    
				    
	\end{center}
	{\large 2. prosince 2022 \hfill
		\begin{tabular}{c |c}
			Petr Bartoš \,--\, xbarto0g (vedoucí)\; & \; 40\,\% \\
			Tomáš Rajsigl \,--\, xrajsi01\;         & \; 30\,\% \\
			Lukáš Zedek \,--\, xzedek03\;           & \; 30\,\% \\
			Dmytro Afanasiev \,--\, xafana01\;        & \; 0\,\%  \\
		\end{tabular}
	}
\end{titlepage}

\section{Úvod}
Cílem projektu je vytvořit program v jazyce C, který načte zdrojový kód zapsaný ve zdrojovém jazyce IFJ22
a přeloží jej do cílového jazyka IFJcode22. Jazyk IFJ22 je zjednodušenou podmnožinou jazyka PHP. Konkrétně se jedná o variantu zadání s implementací tabulky symbolů pomocí tabulky s rozptýlenými položkami.

\section{Práce v týmu}
Práci jsme si rozdělili jakožto čtyřčlenný tým, avšak z důvodu nespolupracování a nezájmu jednoho člena jsme byli nuceni si práci přerozdělit a projekt vypracovat pouze ve třech. Vzhledem k časové náročnosti a složitosti daných částí jsme se rozhodli pro nerovnoměrné rozdělení bodů.

\subsection{Rozdělení práce}

\begin{itemize}
	\item Petr Bartoš
	      \begin{itemize}
	      	\item Syntaktická a sémantická analýza, kostra pro testovaní, tabulka symbolů, Makefile
	      \end{itemize}
	\item Tomáš Rajsigl 
	      \begin{itemize}
	      	\item Lexikální analýza, dokumentace
	      \end{itemize}
	\item Lukáš Zedek
	      \begin{itemize}
	      	\item Generování cílového kódu, testování, dokumentace
	      \end{itemize}
\end{itemize}
Kontrole kódu, jeho čitelnosti a opravě chyb jsme se věnovali všichni.
\subsection{Vývojový cyklus}
Při vypracovávání projektu jsme využívali verzovací systém Git. Pro dané části projektu byly vytvořeny konkrétní branche, kde jsme je testovali a upravovali. Před zahrnutím do master branche byl vyžadován pull request, následný code review a schválení od jednoho či více členů. Součástí vývojového cyklu byl i unit testing. Při každém commitu byly automaticky spuštěny testy a bylo tak hned možné vidět, jaký dopad bude commit mít na výsledný program.

\section{Návrh a implementace překladače}
Projekt byl rozdělen na několik konrétních částí, které jsou představeny v této kapitole.
\clearpage

\subsection{Lexikální analýza}

\subsubsection{Dynamický řetězec}

\subsection{Syntaktická analýza}

\subsubsection{Precedenční syntaktická analýza}

\clearpage

\subsection{Sémantická analýza}

\subsection{Generování cílového kódu}

\clearpage

\subsection{Přílohy}

\subsubsection{Diagram konečného automatu}
    
\clearpage

\subsubsection{LL\,--\,gramatika}
\begin{table}[!ht]
	\centering
	\begin{enumerate}[noitemsep]
		\item \verb|<program>   -> <body_main> <program_e>|
		      		         
		\item \verb|<program_e> -> |$\varepsilon$
		\item \verb|<program_e> -> ?>|
		      		         
		\item \verb|<type_p>    -> float|
		\item \verb|<type_p>    -> int|
		\item \verb|<type_p>    -> string|
		\item \verb|<type>      -> null|
		\item \verb|<type>      -> <type_p>|
		\item \verb|<type>      -> ?<type_p>|
		      		      
		\item \verb|<return>    -> return <return_p>|
		\item \verb|<return_p>  -> expr|
		\item \verb|<return_p>  -> |$\varepsilon$
		      		         
		\item \verb|<body_main> -> <body>|
		\item \verb|<body_main> -> <func_def> <body>|
		      		         
		\item \verb|<body>      -> |$\varepsilon$
		\item \verb|<body>      -> <if> ; <body>|
		\item \verb|<body>      -> <while> ; <body>|
		\item \verb|<body>      -> expr ; <body>|
		\item \verb|<body>      -> identifier_var = expr ; <body>|
		\item \verb|<body>      -> indentifier_var = <func> ; <body>|
		\item \verb|<body>      -> <func> ; <body>|
		\item \verb|<body>      -> <return> ; <body>|
		      		         
		\item \verb|<if>        -> if expr { <body> } else { <body> }|
		\item \verb|<while>     -> while expr { <body> }|
		      		         
		\item \verb|<params>    -> |$\varepsilon$
		\item \verb|<params>    -> <type> identifier_var <params_n>|
		\item \verb|<params>    -> identifier_var <params_n>|
		\item \verb|<params>    -> expr <params_n>|
		\item \verb|<params_n>  -> |$\varepsilon$
		\item \verb|<params_n>  -> , <params_p> <params_n>|
		      		      
		\item \verb|<params_p>  -> <type> identifier_var|
		\item \verb|<params_p>  -> identifier_var|
		\item \verb|<params_p>  -> expr|
		      		         
		\item \verb|<func_def>  -> function identifier_func ( <params> ) : <type> { <body> }|
		\item \verb|<func>      -> identifier_func ( <params> )|
		      		         
	\end{enumerate}
	\caption{LL\,--\,gramatika řídící syntaktickou analýzu}
	\label{table:ll-gramatika}
\end{table}
\clearpage
  
\subsubsection{LL\,--\,tabulka}

\subsubsection{Precedenční tabulka}

\section{Reference}



\end{document}