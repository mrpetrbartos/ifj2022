\documentclass[a4paper,12pt]{article}
\usepackage[utf8]{inputenc}
\usepackage[IL2]{fontenc}
\usepackage[left=1.5cm,top=2.5cm,text={18cm,25cm}]{geometry}
\usepackage[czech]{babel}
\usepackage{times}
\usepackage{enumitem}
\usepackage{array,etoolbox}
\preto\tabular{\setcounter{rule_table_number}{0}}
\newcounter{rule_table_number}
\newcommand\rownumber{\stepcounter{rule_table_number}\arabic{rule_table_number}}
\usepackage{graphics}
\usepackage{pict2e}
\usepackage{svg}
\usepackage{pdflscape}
\usepackage{csquotes}
\usepackage[unicode,pdftitle={Překladač jazyka IFJ22},hidelinks]{hyperref}
\usepackage[backend=biber,style=iso-numeric,autolang=other,sortlocale=$cs_CZ$,bibencoding=UTF8,block=ragged]{biblatex}
\addbibresource{bibliography.bib}
\usepackage{bigfoot}

\renewcommand{\thempfootnote}{\arabic{mpfootnote}}

\begin{document}
\begin{titlepage}
	\begin{center}
		\textsc{{\Huge Vysoké učení technické v~Brně\\[0.4em]}}
		\textsc{{\huge Fakulta informačních technologií}}
		\vspace{\stretch{0.382}}
																										    
		{\Large Dokumentace projektu z~předmětů IFJ a IAL\\[0.3em]}
		\textbf{{\LARGE Implementace překladače imperativního jazyka IFJ22\\[0.4em]}}
		{\Large Tým xbarto0g, varianta\,--\,TRP}
		\vspace{\stretch{0.618}}
																										    
																										    
	\end{center}
	{\large 2. prosince 2022 \hfill
		\begin{tabular}{c |c}
			Petr Bartoš \,--\, xbarto0g (vedoucí)\; & \; 40\,\% \\
			Tomáš Rajsigl \,--\, xrajsi01\;         & \; 30\,\% \\
			Lukáš Zedek \,--\, xzedek03\;           & \; 30\,\% \\
			Dmytro Afanasiev \,--\, xafana01\;        & \; 0\,\%  \\
		\end{tabular}
	}
\end{titlepage}

\section{Úvod}
Cílem projektu je vytvořit program v~jazyce C, který načte zdrojový kód zapsaný ve zdrojovém jazyce IFJ22
a přeloží jej do cílového jazyka IFJcode22. Jazyk IFJ22 je zjednodušenou podmnožinou jazyka PHP.

V~následující dokumentaci je postup řešení tohoto projektu a popis implementace jeho jednotlivých částí.

\section{Práce v~týmu}
Práci jsme si rozdělili jakožto čtyřčlenný tým, avšak z~důvodu nespolupracování a nezájmu jednoho člena jsme byli nuceni si práci přerozdělit a projekt vypracovat pouze ve třech. Vzhledem k~časové náročnosti a složitosti daných částí jsme se rozhodli pro nerovnoměrné rozdělení bodů.

\subsection{Rozdělení práce}

\begin{itemize}
	\item Petr Bartoš
	      \begin{itemize}
	      	\item Návrh gramatik pro syntaktickou analýzu, syntaktická analýza a  sémantická analýza, tabulka a zásobník symbolů,  chybové hlášky, kostra pro testování
	      \end{itemize}
	\item Tomáš Rajsigl 
	      \begin{itemize}
	      	\item Návrh konečného automatu a gramatik pro syntaktickou analýzu, lexikální analýza, testování jednotlivých částí překladače, dokumentace
	      \end{itemize}
	\item Lukáš Zedek
	      \begin{itemize}
	      	\item Generování cílového kódu, testování jednotlivých částí překladače, Makefile, dokumentace a prezentace
	      \end{itemize}
\end{itemize}
Kontrole kódu, jeho čitelnosti a opravě chyb jsme se věnovali všichni tři.
\subsection{Vývojový cyklus}
Při vypracovávání projektu jsme využívali verzovací systém Git. Zdrojové kódy jsme měli uloženy v~repozitáři webové služby GitHub, díky čemuž jsme měli všichni vždy přístup k~nejnovější verzi projektu. 

Pro dané části projektu byly vytvořeny konkrétní větve, kde jsme je testovali a upravovali. Před zahrnutím do hlavní branche byl vyžadován pull request, následný code review a schválení od jednoho či více členů. 

Součástí vývojového cyklu byl i unit testing. Při každém commitu byly automaticky spuštěny testy a bylo tak hned možné vidět, jaký dopad bude commit mít na výsledný program.
\clearpage

\section{Návrh a implementace překladače}
Projekt byl rozdělen na několik konkrétních částí, které jsou postupně představeny v~této kapitole.

\subsection{Lexikální analýza}
Prvním krokem překladače byl návrh a následná implementace lexikální analýzy. Celý lexikální analyzátor je implementován jako deterministický konečný automat, jehož diagram lze nalézt v~obrázku \ref{fig:DKA}. Řešení se nachází ve zdrojovém souboru \verb|scanner.c| a hlavičkovém souboru \verb|scanner.h|.

Hlavní funkce tohoto lexikálního analyzátoru je \verb|getToken|, pomocí které se čte znak po znaku ze zdrojového souboru a převádí se na strukturu \verb|Token|, která se skládá z~typu, hodnoty a z~důvodu vypisování chybových hlášek\footnote{K nalezení v~souborech \verb|error.{c,h}|.} také pozice. Typy tokenu jsou \verb|EOF|, prázdný token, identifikátory, klíčová slova, celé či desetinné číslo, řetězec a~také porovnávací a~aritmetické operátory a~ostatní znaky, které mohou být použity v~jazyce IFJ22. Hodnota tokenu je typu \verb|union| a přiděluje se tokenům \verb|TOKEN_KEYWORD|, \verb|TOKEN_INT|, \verb|TOKEN_FLOAT| a \verb|TOKEN_STRING|.

V~ojedinělých případech je nutno nahlédnout na další token\footnote{Tj. získat stejný token při více voláních.}, v~takové situaci je nastaven \verb|bool| \verb|peek| na hodnotu \verb|true|. Token je poté vracen do prvního volání s~parametrem \verb|peek| nastaveným na \verb|false|.
 
Ve zdrojovém souboru se jedná o~dlouhý \verb|switch|, kde každý případ \verb|case| je ekvivalentní k~jednomu stavu automatu. Pokud načtený znak nesouhlasí s~žádným znakem, který jazyk povoluje, program je ukončen a~vrací chybu 1. Jinak se přechází do dalších stavů a~načítají se další znaky, dokud nemáme hotový jeden token, tj. dokud nedostaneme \verb|tokenComplete| \verb|=| \verb|true| a token potom vracíme a~ukončíme tuto funkci. 

Pro zpracovávání hexadecimálních a oktálních escape sekvencí v~řetězci máme vytvořena dvě pole o~velikosti 3 a 4, která jsou zpočátku vynulována a poté se postupně
naplňují přečtenými čísly, vždy na pozici podle toho, kolikáté číslo zrovna uvažujeme a nakonec celé číslo převedeme do ASCII podoby.

\subsubsection{Prolog a epilog}
\label{prolog}
Na zpracování prologu jsme se rozhodli vytvořit funkci \verb|initialScan|, která v~lexikálním analyzátoru kontroluje první část, a to otevírací značku \verb|<?php|, druhou část -- \verb|declare(strict_types=1);| posíláme jakožto jednotlivé tokeny dále syntaktické analýze, ve které provedeme dodatečnou kontrolu.

Pro zavírací značku volitelného epilogu je vytvořen \verb|TOKEN_CLOSING_TAG|, za kterým je očekáván \verb|EOF| a nebo \verb|\n| a poté \verb|EOF|, jinak vrací chybu 1.

\subsubsection{Řetězec s~proměnnou délkou}
K~zpracovávání identifikátorů a~klíčových slov používáme vlastní strukturu \verb|vStr|, která je implementována ve zdrojovém souboru \verb|vstr.c| s~příslušným hlavičkovým souborem \verb|vstr.h|.

Jedná se o~pole znaků, které se dynamicky zvětšuje v~závislosti na své délce -- při dosáhnutí aktuální maximální délky řetězce je délka zdvojnásobena. 

Znaky načítáme a~průběžně ukládáme do dynamického pole a~jakmile odcházíme ze stavu identifikátoru, porovnáváme, jestli načtený řetězec není shodný s~nějakým z~klíčových slov a~podle toho se rozhodneme, zda vracíme typ a hodnotu tokenu jakožto klíčové slovo či identifikátor.
\clearpage

\subsection{Syntaktická analýza}
Syntaktická analýza, kterou lze nalézt v~souborech \verb|parser.c| a \verb|parser.h|, vychazí z~LL\,--\,gramatiky, kterou je možno vidět v~tabulce \ref{table:ll-gramatika}. Kvůli dvojí možnosti interpretace tokenu \verb|identifier_var| nejsme schopni na základě LL(1) gramatiky rozhodnout, jakým způsobem bude redukován. Byli jsme tudíž nuceni zvolit LL(2) gramatiku, pomocí které můžeme zjistit zdali se jedná o~přiřazení, nebo o~výraz.

V~souboru \verb|parser.h| je vytvořena struktura \verb|Parser|, ve které jsou data používaná jak pro syntaktickou, tak sémantickou analýzu. Struktura se skládá z~aktuálního tokenu \verb|Token| \verb|currToken|, ukazatele na tabulku symbolů při vnoření do funkce \verb|Symtable| \verb|*localSymtable|, ukazatele na tabulku symbolů pro hlavní tělo programu \verb|Symtable| \verb|*symtable|, ukazatele na zásobník \verb|Stack| \verb|*stack|, značky pro potenciálně nedefinovanou proměnnou \verb|bool| \verb|condDec| a značky \verb|bool| \verb|outsideBody| pro situaci kdy je prováděna instrukce mimo hlavní tělo programu.

Syntaktický analyzátor postupně od lexikálního analyzátoru získává jednotlivé tokeny, na základě kterých je vybráno derivační pravidlo. Neterminály tvořené těmito tokeny jsou poté rozkládány na primitivnější neterminály, dokud nejsou zcela rozloženy na terminály. Tento rozklad probíhá pomocí rekurzivního volání jednotlivých funkcí pro derivační pravidla. Tyto funkce naleznutelné ve zdrojovém souboru jsou například \verb|parseWhile|, \verb|parseIf| nebo \verb|parseBody| a další.

Jak již bylo dříve zmíněno v~sekci \ref{prolog} funkcí \verb|checkPrologue| probíhá v~syntaktickém analyzátoru kontrola druhé části prologu, kde ze lexikálního analyzátoru převzaté tokeny jednotlivě kontrolujeme oproti správnému formátu.

\subsubsection{Precedenční syntaktická analýza}
Výrazy se zpracovávají odděleně od zpracování syntaxe a to v~souborech \verb|expression.{c,h}|. Algoritmus pro zpracování výrazů byl implementován v~souladu s~materiály k~předmětu IFJ, tedy metodou zdola nahoru řídící se dle precedenční tabulky. V~přílohách se jedná o~tabulku \ref{table:prec_table}, podle které může nastat 5 stavů a to: \verb|reduce|, \verb|shift|, \verb|equal|, \verb|error| a \verb|ok|.

Funkce \verb|parseExpression| postupně načítá tokeny do zásobníku, určuje jejich typ a pomocí tabulky určuje vztah mezi nejvyšším terminálem na zásobníku a tokenem na vstupu. Pokud žádné pravidlo neodpovídá situaci na vrcholu zásobníku, tak je program ukončen a vrací chybu 2.


\subsubsection{Tabulka symbolů}
Tabulku symbolů\footnote{V souborech \verb|symtable.{c,h}|.} jsme implementovali jako tabulku s~rozptýlenými položkami (hashovací tabulku). Každý index pole ukazuje na lineárně vázaný seznam prvků, které jsou přidávány do tabulky zepředu tak, že nejstarší prvky na daném indexu jsou poslední v~seznamu. Klíčem do hashovací tabulky jsou samotné názvy identifikátorů.

Každý prvek tabulky kromě klíče a~ukazatele na následující prvek obsahuje typ, data pro funkci a~data pro proměnnou. Data pro funkci obsahují počet parametrů a~ukazatel na lineárně vázaný seznam parametrů.

Jelikož jeden z~členů týmu úspěšně absolvoval předmět IJC, rozhodli jsme se využít hashovací funkci a strukturu tabulky symbolů z~2. DÚ tohoto předmětu\cite{IJC2DU}.

\clearpage

\subsubsection{Zásobník symbolů}
Pro syntaktickou a sémantickou analýzu jsme implementovali zásobník v~souboru \verb|structures.c|, jehož rozhraní je v~souboru \verb|structures.h|. Má implementovány základní zásobníkové operace jako \verb|stackPush|, \verb|stackPop| a \verb|stackFree|.

V~úvahu by přišla záměna zásobníku za dvousměrně vázaný seznam kvůli jednoduchosti vkládání zarážek. V~případě implementace se zásobníkem je třeba „překážející“ hodnoty dočasně uložit na jiný zásobník a poté vracet zpět, což je neefektivní.

\subsection{Sémantická analýza}
Sémantická analýza je prováděna podle sémantických pravidel jazyka IFJ22. Z~části se vykonává zároveň se syntaktickou analýzou  a z~části se vykonává až po skončení syntaktické analýzy -- tudíž za běhu programu. Taktéž používá strukturu \verb|Parser|, ve které jsou potřebná data pro sémantické kontroly. 

Na příslušných místech v~souboru \verb|parser.c| jsou tedy prováděny kontroly zda jsou používané funkce a proměnné definované -- pokud ne, tak se pro funkce jedná o~chybu 3 a pro proměnné o~chybu 5. Dále jestli odpovídá počet/typ parametrů u~volání funkce či typ návratové hodnoty z~funkce, což by v~případě nesplnění byla chyba 4.

Další sémantická kontrola se provádí za běhu programu. V~souboru \verb|codegen.c| kontrolujeme typovou kompatibilitu při operacích s~výrazy. Při takovýchto\footnote{Např. jeden operand typu float a druhý typu int.} operacích musíme případně operandy přetypovat.

\subsection{Generování cílového kódu}
Generování výstupního kódu IFJcode22 je poslední částí našeho překladače. Generátor \verb|codegen.c| se skládá z~několika funkcí, volaných z~náležité pozice v~syntaktickém analyzátoru. 

Při generování samotného cílového kódu se jako první generuje hlavička, sestávájící ze sekvence tří částí, kdy první část \verb|genPrintHead| obsahuje globální pomocné proměnné a vytváří lokální rámec pro hlavní tělo programu, druhá část \verb|genCheckTruth| obsahující znovupoužitelnou funkci pro vyhodnocení pravdivostní hodnoty libovolného literálu a třetí část \verb|genMathInstCheck|, jež obstarává již zmiňovanou typovou kontrolu a případnou konverzi vstupních proměnných pro aritmetické instrukce\footnote{Příkladem mohou být instrukce \verb|ADDS|, \verb|SUBS|, \verb|MULS|.}.

Kód následující hlavičce je poté generován dle předaných tokenů ze syntaktického analyzátoru. Generování kódu funkcí definovaných uživatelem je možné díky univerzálním funkcím \verb|genFuncDef| definující tělo funkce a \verb|genFuncCall|, která tyto funkce následně volá. Obdobně funguje funkce \verb|genIfElse| definující tělo podmíněného příkazu. Generování vestavěných funkcí jazyka IFJ22 je implementováno pro každou z~nich samostatně.

Speciální pozornost vyžadoval výpis řetězců, jejichž formát musel být změněn na formát předepsaný jazykem IFJcode22. Změna formátu probíhá ve funkci \verb|genStackPush|, která se mimo jiné stará o~přesun hodnot tokenů na zásobník.

\clearpage

\section{Přílohy}

\subsection{Diagram konečného automatu}

\begin{figure}[htp!]
	\centerline{\includesvg[inkscapelatex=false,width=1.05\columnwidth]{Los_Angeles.svg}}
	\begin{minipage}{\textwidth}
		\vspace{1.5em}
		\caption{Konečný automat\protect\footnote[6]{Z důvodu úspory místa a čitelnosti byly jednotlivé stavy přejmenovány, legenda stavů následuje na další straně.} specifikující lexikální analyzátor}
		\label{fig:DKA}
	\end{minipage}
\end{figure}

\clearpage

\subsubsection{Legenda}
\vspace{1.5em}
\begin{center}
				    
	\begin{tabular}{l r|l r}
		\verb|S|    & \; \verb|STATE_INITIAL|    \;      & \; \verb|/|       & \;  \verb|STATE_SLASH|             \\
		\verb|ID|   & \; \verb|STATE_IDENTIFIER|  \;     & \; \verb|L_COM|   & \;  \verb|STATE_LINE_COM|          \\
		\verb|+|    & \; \verb|STATE_PLUS|        \;     & \;  \verb|M_COM|  & \;  \verb|STATE_MULTILINE_COM|     \\
		\verb|-|    & \; \verb|STATE_MINUS|       \;     & \;  \verb|M_END|  & \;  \verb|STATE_POT_MULTILINE_END| \\
		\verb|*|    & \; \verb|STATE_MULTIPLY|      \;   & \;  \verb|STR_0|  & \;  \verb|STATE_STRING_0|          \\
		\verb|.|    & \; \verb|STATE_CONCATENATE|  \;    & \;  \verb|STR_1|  & \;  \verb|STATE_STRING_1|          \\
		\verb|=|    & \; \verb|STATE_EQUAL_OR_ASSIGN| \; & \; \verb|ESC_SQ|  & \;  \verb|STATE_STRING_ESCAPE|     \\
		\verb|==|   & \; \verb|STATE_EQUAL_0|       \;   & \;  \verb|HEX_0|  & \;  \verb|STATE_STRING_HEXA_0|     \\
		\verb|===|  & \; \verb|STATE_EQUAL_1|       \;   & \;   \verb|HEX_1| & \;  \verb|STATE_STRING_HEXA_1|     \\
		\verb|<|    & \; \verb|STATE_LESSTHAN|      \;   & \;   \verb|OCT_0| & \;  \verb|STATE_STRING_OCTA_0|     \\
		\verb|>|    & \; \verb|STATE_GREATERTHAN|    \;  & \;  \verb|OCT_1|  & \;  \verb|STATE_STRING_OCTA_1|     \\
		\verb|(|    & \; \verb|STATE_LEFT_BRACKET|   \;  &                   &                                    \\
		\verb|)|    & \; \verb|STATE_RIGHT_BRACKET|  \;  &                   &                                    \\
		\verb|{|    & \; \verb|STATE_LEFT_BRACE|     \;  &                   &                                    \\
		\verb|}|    & \; \verb|STATE_RIGHT_BRACE|    \;  &                   &                                    \\
		\verb|EOF|  & \; \verb|STATE_EOF|           \;   &                   &                                    \\
		\verb|?|    & \; \verb|STATE_OPTIONAL|      \;   &                   &                                    \\
		\verb|?>_0| & \verb|STATE_CLOSING_TAG_0|    \;   &                   &                                    \\
		\verb|?>_1| & \verb|STATE_CLOSING_TAG_1|    \;   &                   &                                    \\
		\verb|?>|   & \verb|STATE_CLOSING_TAG_2|    \;   &                   &                                    \\
		\verb|,|    & \verb|STATE_COMMA|            \;   &                   &                                    \\
		\verb|:|    & \verb|STATE_COLON|            \;   &                   &                                    \\
		\verb|;|    & \verb|STATE_SEMICOLON|       \;    &                   &                                    \\
		\verb|!|    & \verb|STATE_NOT_EQUAL_0|     \;    &                   &                                    \\
		\verb|!=|   & \verb|STATE_NOT_EQUAL_1|      \;   &                   &                                    \\
		\verb|!==|  & \verb|STATE_NOT_EQUAL_2|     \;    &                   &                                    \\
		\verb|$|    & \verb|STATE_VARID_PREFIX|    \;    &                   &                                    \\
		\verb|INT|  & \verb|STATE_NUMBER|           \;   &                   &                                    \\
		\verb|E_0|  & \verb|STATE_FLOAT_E_0|       \;    &                   &                                    \\
		\verb|E_1|  & \verb|STATE_FLOAT_E_1|       \;    &                   &                                    \\
		\verb|E_2|  & \verb|STATE_FLOAT_E_2|       \;    &                   &                                    \\
		\verb|FL_0| & \verb|STATE_FLOAT_0|        \;     &                   &                                    \\
		\verb|FL_1| & \verb|STATE_FLOAT_1|         \;    &                   &                                    \\
	\end{tabular}
\end{center}
\clearpage

\subsection{LL\,--\,gramatika}
\begin{table}[!ht]
	\centering
	\begin{enumerate}[noitemsep]
		\item \verb|<program>    -> <body_main> <program_e>|
		      		      		      		      		      		      		      		      		      	   
		\item \verb|<program_e>  -> |$\varepsilon$
		\item \verb|<program_e>  -> ?>|
		      		      		      		      		      		      		    		      		      		    
		\item \verb|<type_p>     -> float|
		\item \verb|<type_p>     -> int|
		\item \verb|<type_p>     -> string|
		\item \verb|<type_n>     -> ? <type_p>|
		\item \verb|<type>       -> void|
		\item \verb|<type>       -> <type_p>|
		\item \verb|<type>       -> <type_n>|
		      		      		      		      		      		      		      	      		      	      
		\item \verb|<return>     -> return <return_p>|
		\item \verb|<return_p>   -> expr|
		\item \verb|<return_p>   -> |$\varepsilon$
		      		      		      		      		      		      		      	      		      		    
		\item \verb|<body_main>  -> <body> <body_main>|
		\item \verb|<body_main>  -> <func_def> <body_main>|
		\item \verb|<body_main>  -> |$\varepsilon$
		      		      		      		      		      		      		      		      		      	    
		\item \verb|<body>       -> <if>|
		\item \verb|<body>       -> <while>|
		\item \verb|<body>       -> expr ;|
		\item \verb|<body>       -> identifier_var = <assign_v> ;|
		\item \verb|<body>       -> <func> ;|
		\item \verb|<body>       -> <return> ;|
		      		      		      		      		      		      		              
		\item \verb|<assign_v>   -> expr|
		\item \verb|<assign_v>   -> <func>|
		      		      		      		      		      		      		              
		\item \verb|<pe_body>    -> <body> <pe_body>|
		\item \verb|<pe_body>    -> |$\varepsilon$
		      		      		      		      		      		      		      		      		      	    
		\item \verb|<if>         -> if ( expr ) { <pe_body> } else { <pe_body> }|
		\item \verb|<while>      -> while ( expr ) { <pe_body> }|
		      		      		      		      		      		      		      		      		    	    
		\item \verb|<params_dn>  -> |$\varepsilon$
		\item \verb|<params_dn>  -> , <type_n> identifier_var <params_dn>|
		      		      		      		      		      		      		              
		\item \verb|<params_dp>  -> <type_n> identifier_var <params_dn>|
		\item \verb|<params_dp>  -> |$\varepsilon$
		      		      		      		      		      
		\item \verb|<params_cnp> -> lit|
		\item \verb|<params_cnp> -> identifier_var|
		      		      		      		      		      		      		              
		\item \verb|<params_cn>  -> |$\varepsilon$
		\item \verb|<params_cn>  -> , <params_cnp> <params_cn>|
		      		      		         		      		              
		\item \verb|<params_cp>  -> lit <params_cn>|
		\item \verb|<params_cp>  -> identifier_var <params_cn>|
		\item \verb|<params_cp>  -> |$\varepsilon$
		      		      		      		      		      		      		      		      		      	    
		\item \verb|<func_def>   -> function identifier_func ( <params_dp> ) :| 
		      \newline \verb |               <type> { <pe_body> }|
		\item \verb|<func>       -> identifier_func ( <params_cp> )|
		      		      		      		      		      		      		      		      		      	    
	\end{enumerate}
	\caption{LL\,--\,gramatika řídící syntaktickou analýzu}
	\label{table:ll-gramatika}
\end{table}
\clearpage

\begin{landscape}
	\subsection{LL\,--\,tabulka}
	\begin{table}[!ht]
		\resizebox{\columnwidth}{!}{
			\centering
			\begin{tabular}{|c|c|c|c|c|c|c|c|c|c|c|c|c|c|c|c|c|c|c|c|c|c|c|c|c|}
				\hline
				\textbf{}                                     & \textbf{?\textgreater{}} & \textbf{float} & \textbf{int} & \textbf{string} & \textbf{?} & \textbf{void} & \textbf{return} & \textbf{expr} & \textbf{;} & \textbf{identifier\_var}                    & \textbf{=} & \textbf{if} & \textbf{(} & \textbf{)} & \textbf{\{} & \textbf{\}} & \textbf{else} & \textbf{while} & \textbf{,} & \textbf{lit}                               & \textbf{function} & \textbf{identifier\_func} & \textbf{:} & \textbf{\$} \\ \hline
				\textbf{\textless{}program\textgreater{}}     & 1                        & ~              & ~            & ~               & ~          & ~             & 1               & 1             & ~          & 1                                           & ~ & 1  & ~ & ~  & ~ & ~  & ~ & 1  & ~  & ~                                          & 1  & 1  & ~ & 1  \\ \hline
				\textbf{\textless{}program\_e\textgreater{}}  & 3                        & ~              & ~            & ~               & ~          & ~             & ~               & ~             & ~          & ~                                           & ~ & ~  & ~ & ~  & ~ & ~  & ~ & ~  & ~  & ~                                          & ~  & ~  & ~ & 2  \\ \hline
				\textbf{\textless{}type\_p\textgreater{}}     & ~                        & 4              & 5            & 6               & ~          & ~             & ~               & ~             & ~          & ~                                           & ~ & ~  & ~ & ~  & ~ & ~  & ~ & ~  & ~  & ~                                          & ~  & ~  & ~ & ~  \\ \hline
				\textbf{\textless{}type\_n\textgreater{}}     & ~                        & ~              & ~            & ~               & 7          & ~             & ~               & ~             & ~          & ~                                           & ~ & ~  & ~ & ~  & ~ & ~  & ~ & ~  & ~  & ~                                          & ~  & ~  & ~ & ~  \\ \hline
				\textbf{\textless{}type\textgreater{}}        & ~                        & 9              & 9            & 9               & 10         & 8             & ~               & ~             & ~          & ~                                           & ~ & ~  & ~ & ~  & ~ & ~  & ~ & ~  & ~  & ~                                          & ~  & ~  & ~ & ~  \\ \hline
				\textbf{\textless{}return\textgreater{}}      & ~                        & ~              & ~            & ~               & ~          & ~             & 11              & ~             & ~          & ~                                           & ~ & ~  & ~ & ~  & ~ & ~  & ~ & ~  & ~  & ~                                          & ~  & ~  & ~ & ~  \\ \hline
				\textbf{\textless{}return\_p\textgreater{}}   & ~                        & ~              & ~            & ~               & ~          & ~             & ~               & 12            & 13         & ~                                           & ~ & ~  & ~ & ~  & ~ & ~  & ~ & ~  & ~  & ~                                          & ~  & ~  & ~ & ~  \\ \hline
				\textbf{\textless{}body\_main\textgreater{}}  & 16                       & ~              & ~            & ~               & ~          & ~             & 14              & 14            & ~          & 14                                          & ~ & 14 & ~ & ~  & ~ & ~  & ~ & 14 & ~  & ~                                          & 15 & 14 & ~ & 16 \\ \hline
				\textbf{\textless{}body\textgreater{}}        & ~                        & ~              & ~            & ~               & ~          & ~             & 22              & 19            & ~          & 20                                          & ~ & 17 & ~ & ~  & ~ & ~  & ~ & 18 & ~  & ~                                          & ~  & 21 & ~ & ~  \\ \hline
				\textbf{\textless{}assign\_v\textgreater{}}   & ~                        & ~              & ~            & ~               & ~          & ~             & ~               & 23            & ~          & ~                                           & ~ & ~  & ~ & ~  & ~ & ~  & ~ & ~  & ~  & ~                                          & ~  & 24 & ~ & ~  \\ \hline
				\textbf{\textless{}pe\_body\textgreater{}}    & ~                        & ~              & ~            & ~               & ~          & ~             & 25              & 25            & ~          & 25                                          & ~ & 25 & ~ & ~  & ~ & 26 & ~ & 25 & ~  & ~                                          & ~  & 25 & ~ & ~  \\ \hline
				\textbf{\textless{}if\textgreater{}}          & ~                        & ~              & ~            & ~               & ~          & ~             & ~               & ~             & ~          & ~                                           & ~ & 27 & ~ & ~  & ~ & ~  & ~ & ~  & ~  & ~                                          & ~  & ~  & ~ & ~  \\ \hline
				\textbf{\textless{}while\textgreater{}}       & ~                        & ~              & ~            & ~               & ~          & ~             & ~               & ~             & ~          & ~                                           & ~ & ~  & ~ & ~  & ~ & ~  & ~ & 28 & ~  & ~                                          & ~  & ~  & ~ & ~  \\ \hline
				\textbf{\textless{}params\_dn\textgreater{}}  & ~                        & ~              & ~            & ~               & ~          & ~             & ~               & ~             & ~          & ~                                           & ~ & ~  & ~ & 29 & ~ & ~  & ~ & ~  & 30 & ~                                          & ~  & ~  & ~ & ~  \\ \hline
				\textbf{\textless{}params\_dp\textgreater{}}  & ~                        & ~              & ~            & ~               & 31         & ~             & ~               & ~             & ~          & ~                                           & ~ & ~  & ~ & 32 & ~ & ~  & ~ & ~  & ~  & ~                                          & ~  & ~  & ~ & ~  \\ \hline
				\textbf{\textless{}params\_cnp\textgreater{}} & ~                        & ~              & ~            & ~               & ~          & ~             & ~               & ~             & ~          & 34                                          & ~ & ~  & ~ & ~  & ~ & ~  & ~ & ~  & ~  & 33                                         & ~  & ~  & ~ & ~  \\ \hline
				\textbf{\textless{}params\_cn\textgreater{}}  & ~                        & ~              & ~            & ~               & ~          & ~             & ~               & ~             & ~          & ~                                           & ~ & ~  & ~ & 35 & ~ & ~  & ~ & ~  & 36 & ~                                          & ~  & ~  & ~ & ~  \\ \hline
				\textbf{\textless{}params\_cp\textgreater{}}  & ~                        & ~              & ~            & ~               & ~          & ~             & ~               & ~             & ~          & 38                                          & ~ & ~  & ~ & 39 & ~ & ~  & ~ & ~  & ~  & 37                                         & ~  & ~  & ~ & ~  \\ \hline
				\textbf{\textless{}func\_def\textgreater{}}   & ~                        & ~              & ~            & ~               & ~          & ~             & ~               & ~             & ~          & ~                                           & ~ & ~  & ~ & ~  & ~ & ~  & ~ & ~  & ~  & ~                                          & 40 & ~  & ~ & ~  \\ \hline
				\textbf{\textless{}func\textgreater{}}        & ~                        & ~              & ~            & ~               & ~          & ~             & ~               & ~             & ~          & ~                                           & ~ & ~  & ~ & ~  & ~ & ~  & ~ & ~  & ~  & ~                                          & ~  & 41 & ~ & ~  \\ \hline
			\end{tabular}
		}
		\vspace{0.75em}
		\caption{LL\,--\,tabulka použitá při syntaktické analýze}
		\label{table:ll_table}
	\end{table}
\end{landscape}

\subsection{Precedenční tabulka}

\begin{table}[h]
	\centering
	\begin{tabular}{|c|c|c|c|c|c|c|c|c|} 
		\hline
		                      & \textbf{+ -}   & \textbf{* /}   & \textbf{i}  & \textbf{~\$~}  & \textbf{Rel}   & \textbf{Cmp}   & \textbf{(}~ & \textbf{~)~}   \\ 
		\hline
		\textbf{\textbf{+ -}} & \textgreater{} & \textless{}    & \textless{} & \textgreater{} & \textgreater{} & \textgreater{} & \textless{} & \textgreater{} \\ 
		\hline
		\textbf{\textbf{* /}} & \textgreater{} & \textgreater{} & \textless{} & \textgreater{} & \textgreater{} & \textgreater{} & \textless{} & \textgreater{} \\ 
		\hline
		\textbf{\textbf{i}}   & \textgreater{} & \textgreater{} &             & \textgreater{} & \textgreater{} & \textgreater{} &             & \textgreater{} \\ 
		\hline
		\textbf{\textbf{\$}}  & \textless{}    & \textless{}    & \textless{} &                & \textless{}    & \textless{}    & \textless{} &                \\ 
		\hline
		\textbf{\textbf{Rel}} & \textless{}    & \textless{}    & $<$         & \textgreater{} &                &                & \textless{} & \textgreater{} \\ 
		\hline
		\textbf{\textbf{Cmp}} & \textless{}    & \textless{}    & \textless{} & \textgreater{} &                &                & \textless{} & \textgreater{} \\ 
		\hline
		\textbf{\textbf{(}}   & \textless{}    & \textless{}    & \textless{} &                & \textless{}    & \textless{}    & \textless{} & =              \\ 
		\hline
		\textbf{\textbf{)}}   & \textgreater{} & \textgreater{} &             & \textgreater{} & \textgreater{} & \textgreater{} &             & \textgreater{} \\
		\hline
	\end{tabular}
	\begin{minipage}{\textwidth}
		\vspace{1.5em}
		\caption{Precedenční tabulka\protect\footnote[7]{Rel značí relační operátory a Cmp je zkratka pro komparační operátory.} použitá při precedenční syntaktické analýze výrazů}
		\label{table:prec_table}
	\end{minipage}
\end{table}
\clearpage

\section{Reference}
\nocite{*}
\printbibliography[heading=none]

\end{document}